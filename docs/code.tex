\section{Листинги}

\subsection{Алгоритмы, реализующие бизнес-логику}
\subsubsection{Операции с курсовыми (получение, добавление, удаление, обновление)}
\begin{minted}{cpp}
ObjectsContainer* ViewsCollection::allProjects(char* method, char* requestBody)
{
	if (strcmp(method, "GET") == 0) //GET - значит выборка
		return DatabaseWrapper::instance()->getObjects(new Project());
	if ((strcmp(method, "PUT") == 0)||(strcmp(method, "POST") == 0)) 
	//PUT или POST - значит обновление
	{
		Project *object = new Project(*(Project*)requestBody);

		ObjectsContainer *result; //разбираюсь с внешними ключами:
		result = DatabaseWrapper::instance()->getObjectsByAttribute(new Lecturer(), 
		L"name", object->lecturer); //со ссылкой на преподавателя
		if (result->next() == FALSE) throwAnException("Foreign key error: no such 
		lecturer"); //не вернулось ни одного объекта - что-то пошло не так, 
		бросаю исключение
		Lecturer* related_object = (Lecturer*)(result->current());
		object->lecturer_id = related_object->id;

		result = DatabaseWrapper::instance()->getObjectsByAttribute(new Student(), 
		L"name", object->student); //и на студента
		if (result->next() == FALSE) throwAnException("Foreign key error: no such 
		student");
		Student* another_related_object = (Student*)(result->current());
		object->student_id = another_related_object->id;

		DatabaseWrapper::instance()->updateObject(object);
		return NULL;
	}
	if (strcmp(method, "DELETE") == 0) //DELETE - значит удаление
	{
		DatabaseWrapper::instance()->deleteObject(new Project(*(Project*)requestBody));
		return NULL;
	}
}
\end{minted}

\subsubsection{Выборка курсовых по группе или преподавателю}
\begin{minted}{cpp}
ObjectsContainer* ViewsCollection::groupProjects(char method[6], 
char* requestBody)
{
	return DatabaseWrapper::instance()->getObjectsByAttribute(new Project(), 
	L"student_group", (LPCWSTR)requestBody);
}

ObjectsContainer* ViewsCollection::lecturerProjects(char method[6],
char* requestBody)
{
	return DatabaseWrapper::instance()->getObjectsByAttribute(new Project(), 
	L"lecturer", (LPCWSTR)requestBody);
}
\end{minted}

\subsubsection{Операции со студентами}
\begin{minted}{cpp}
ObjectsContainer* ViewsCollection::allStudents(char method[6], char* requestBody)
{
	if (strcmp(method, "GET") == 0)
		return DatabaseWrapper::instance()->getObjects(new Student());
	if (strcmp(method, "PUT") == 0)
	{
		DatabaseWrapper::instance()->updateObject(new Student(
		*(Student*)requestBody));
		return NULL;
	}
	if (strcmp(method, "POST") == 0)
	{
		DatabaseWrapper::instance()->updateObject(new Student(
		*(Student*)requestBody));
		return NULL;
	}
	if (strcmp(method, "DELETE") == 0)
	{
		DatabaseWrapper::instance()->deleteObject(new Student(
		*(Student*)requestBody));
		return NULL;
	}
}
\end{minted}

\subsubsection{Опреации с преподавателями}
\begin{minted}{cpp}
ObjectsContainer* ViewsCollection::allLecturers(char method[6], char* requestBody)
{
	if (strcmp(method, "GET") == 0)
		return DatabaseWrapper::instance()->getObjects(new Lecturer());
	if (strcmp(method, "PUT") == 0)
	{
		DatabaseWrapper::instance()->updateObject(new Lecturer(
		*(Lecturer*)requestBody));
		return NULL;
	}
	if (strcmp(method, "POST") == 0)
	{
		DatabaseWrapper::instance()->updateObject(new Lecturer(
		*(Lecturer*)requestBody));
		return NULL;
	}
	if (strcmp(method, "DELETE") == 0)
	{
		DatabaseWrapper::instance()->deleteObject(new Lecturer(
		*(Lecturer*)requestBody));
		return NULL;
	}
}
\end{minted}

\subsection{Основные элементы программы}
\subsubsection{Основной цикл сервера}
\begin{minted}{cpp}
while (1) //начинаю обрабатывать запросы
{
	working_socket = accept(listening_socket, (struct sockaddr *)&client_address, 
	&size_of_address); //принимаю подключение клиента

	printf("Client %s connected.\n", inet_ntoa(client_address.sin_addr));
	_beginthread(client_thread, 0, (void*)working_socket);
}
\end{minted}

\subsubsection{Обработка запроса}
\begin{minted}{cpp}
void client_thread(void *socket_to_use)
{
	RequestProcessor *newProcessor = new RequestProcessor((int)socket_to_use); 
	//новый Processor будет обрабатывать запросы
	newProcessor->mainLoopIteration();

	delete newProcessor; //клиент отключился - удаляю Processor, завершаю поток
	closesocket((int)socket_to_use);

	return;
}
\end{minted}
\begin{minted}{cpp}
int RequestProcessor::mainLoopIteration()
{
	while (1)
	{
		RequestHeader *header = new RequestHeader();
		if (recv(working_socket, (char*)header, sizeof(RequestHeader), NULL) >= 0) 
		//принимаю заголовок
		{
			char *body;
			if (header->bodySize > 0)
			{
				body = (char*)malloc(header->bodySize);
				recv(working_socket, body, header->bodySize, NULL); 
				//принимаю тело запроса
			} else {
				body = 0;
			}

			displatchRequest(header, body);
		} else {
			break;
		}
	}

	return 0;
}
\end{minted}

\subsubsection{Выбор функции для выполения запроса}
\begin{minted}{cpp}
void RequestProcessor::displatchRequest(RequestHeader *header, char* body)
{
	char* usecase = header->URI;
	
	printf("%s %s -> ", header->method, header->URI);

	if (strcmp(usecase, "/project")==0) responseDecorator(ViewsCollection::allProjects, 
	header, body); //перебираю представления в поисках подходящего
	if (strcmp(usecase, "/project/lecturer")==0) responseDecorator(
	ViewsCollection::lecturerProjects, header, body);
	if (strcmp(usecase, "/project/group")==0) responseDecorator(
	ViewsCollection::groupProjects, header, body);
	if (strcmp(usecase, "/lecturer")==0) responseDecorator(
	ViewsCollection::allLecturers, header, body);
	if (strcmp(usecase, "/student")==0) responseDecorator(
	ViewsCollection::allStudents, header, body);
}
\end{minted}

\subsubsection{Формирование ответа}
\begin{minted}{cpp}
void RequestProcessor::responseDecorator(viewFunction view, RequestHeader *header,
char* body)
{
	ObjectsContainer *responseBody;
	ResponseHeader *response = new ResponseHeader();
	strcpy(response->URI, header->URI);

	try
	{
		responseBody = view(header->method, body);

		strcpy(response->status, "OK");

		if (responseBody != NULL) //сервер что-то возвращает
		{
			response->bodySize = responseBody->totalSize();
			sendResponse(response, responseBody->dataPointer());
		} else { //сервер не возвращает ничего (значит, и не надо)
			response->bodySize = 0;
			sendResponse(response, NULL);
		}
	} catch (sql::SQLException &error) {
		strcpy(response->status, "FAIL");
		response->bodySize = strlen(error.what())+1;

		sendResponse(response, (char*)error.what());
	}
}
\end{minted}

\subsubsection{Отправка запроса и обработка ответа (на стороне клиента)}
\begin{minted}{cpp}
ResponseBody RequestGenerator::sendRequest(char method[4], char URI[50], char* body, 
int bodySize)
{
	RequestHeader *header = new RequestHeader; //заголовок запроса

	strcpy(header->method, method);
	strcpy(header->URI, URI);
	header->bodySize = bodySize;

	send(socket_id, (char*)header, sizeof(RequestHeader), NULL);
	send(socket_id, body, bodySize, NULL);


	ResponseHeader *response = new ResponseHeader;
	recv(socket_id, (char*)response, sizeof(RequestHeader), NULL); 
	//принимаю заголовок ответа

	char *responseBody = (char*)malloc(response->bodySize);
	if (response->bodySize > 0)
		recv(socket_id, (char*)responseBody, response->bodySize, NULL);

	ResponseBody resp;
	if (strcmp(response->status, "OK") == 0)
	{
		resp.body = responseBody;
		resp.size = response->bodySize;

		return resp;
	} else if (strcmp(response->status, "FAIL") == 0) //если произошла ошибка
	{
		WCHAR *messageEncoded = (WCHAR*)calloc(response->bodySize, sizeof(WCHAR));
		mbstowcs(messageEncoded, responseBody, response->bodySize);
		mainForm->showError(messageEncoded); //показываю сообщение об этом
		
		resp.size = 0;
		return resp; //нуждается а проверке
	} else {
		mainForm->showError(L"Непонятный ответ сервера.");
		
		resp.size = 0;
		return resp;
	}
}
\end{minted}

\subsection{Скрипт генерации базы данных}
\begin{minted}{sql}
CREATE DATABASE IF NOT EXISTS `projects-database`;
USE `projects-database`;

CREATE TABLE IF NOT EXISTS `student` (
  `name` varchar(100) NOT NULL,
  `id` int(10) NOT NULL AUTO_INCREMENT,
  `group` varchar(50) NOT NULL,
  PRIMARY KEY (`id`),
  UNIQUE KEY `name` (`name`)
) DEFAULT CHARSET=utf8;

CREATE TABLE IF NOT EXISTS `lecturer` (
  `name` varchar(100) NOT NULL,
  `id` int(10) NOT NULL AUTO_INCREMENT,
  PRIMARY KEY (`id`),
  UNIQUE KEY `name` (`name`)
) DEFAULT CHARSET=utf8;

CREATE TABLE IF NOT EXISTS `project` (
  `task` varchar(100) NOT NULL,
  `dueTo` date NOT NULL,
  `id` int(11) NOT NULL AUTO_INCREMENT,
  `subject` varchar(50) NOT NULL,
  `completeness` int(10) unsigned NOT NULL DEFAULT '0',
  `student_id` int(10) NOT NULL,
  `lecturer_id` int(10) NOT NULL,
  PRIMARY KEY (`id`),
  KEY `student_ref` (`student_id`),
  KEY `lecturer_ref` (`lecturer_id`),
  CONSTRAINT `lecturer_ref` FOREIGN KEY (`lecturer_id`) REFERENCES `lecturer` (`id`),
  CONSTRAINT `student_ref` FOREIGN KEY (`student_id`) REFERENCES `student` (`id`)
) DEFAULT CHARSET=utf8;
\end{minted}